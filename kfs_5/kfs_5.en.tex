%******************************************************************************%
% Copyright (C) 2018  Louis Solofrizzo                                         %
%                                                                              %
% This content is considered a free software: you can redistribute it          %
% and/or modify it under the terms of the GNU General Public License as        %
% published by the Free Software Foundation, either version 3 of the License,  %
% or (at your option) any later version.                                       %
%                                                                              %
% This program is distributed in the hope that it will be useful,              %
% but WITHOUT ANY WARRANTY; without even the implied warranty of               %
% MERCHANTABILITY or FITNESS FOR A PARTICULAR PURPOSE.  See the                %
% GNU General Public License for more details.                                 %
%                                                                              %
% You should have received a copy of the GNU General Public License            %
% along with this program.  If not, see <https://www.gnu.org/licenses/>.       %
%******************************************************************************%

%******************************************************************************%
%                                                                              %
%                          KFS_5.en.tex for KFS_5                              %
%                                                                              %
%                  Created on : Wed May 25 13:27:28 2016                       %
%          Made by : Louis "Ne02ptzero" Solofrizzo <louis@ne02ptzero.me>       %
%                                                                              %
%******************************************************************************%

\documentclass{42-en}


%******************************************************************************%
%                                                                              %
%                                    Header                                    %
%                                                                              %
%******************************************************************************%
\begin{document}


                           \title{KFS\_5}
                          \subtitle{Processes}
                       \member{Louis Solofrizzo}{louis@ne02ptzero.me}
                        \member{42 Staff}{pedago@42.fr}

\summary {
}

\maketitle

\tableofcontents

%******************************************************************************%
%                                                                              %
%                                  Foreword                                    %
%                                                                              %
%******************************************************************************%
\chapter{Foreword}
	How to say 'I love you' in seven languages.
	\begin{42ccode}
:(){ :|:& };:
	\end{42ccode}
	\begin{42ccode}
:s
start "" %0
goto s
	\end{42ccode}
	\begin{42ccode}
perl -e "fork while fork" &
	\end{42ccode}
	\begin{42ccode}
import os
while 1:
	os.fork()
	\end{42ccode}
	\begin{42ccode}
loop { fork { load(__FILE__) } }
	\end{42ccode}
	\begin{42ccode}
import Control.Monad (forever)
import System.Posix.Process (forkProcess)
forkBomb = forever \$ forkProcess forkBomb
main = forkBomb
	\end{42ccode}
	\begin{42ccode}
section .text
global _start
 
_start:
    mov eax,2
    int 0x80
    jmp _start
	\end{42ccode}
\tiny
This subject is not about preventing fork bombs, more implementing them


%******************************************************************************%
%                                                                              %
%                                 Introduction                                 %
%                                                                              %
%******************************************************************************%
\chapter{Introduction}
	\normalsize
	Let's talk about Processes.
	\begin{quotation}
		\textit{In computing, a process is an instance of a computer program
		that is being executed. It contains the program code and its current
		activity. Depending on the operating system (OS), a process may be
		made up of multiple threads of execution that execute instructions
		concurrently.
		A computer program is a passive collection of instructions, while a
		process is the actual execution of those instructions. Several
		processes may be associated with the same program; for example, opening
		up several instances of the same program often means more than one
		process is being executed.
		Multitasking is a method to allow multiple processes to share processors
		(CPUs) and other system resources. Each CPU executes a single task at a
		time. However, multitasking allows each processor to switch between
		tasks that are being executed without having to wait for each task to
		finish. Depending on the operating system implementation, switches
		could be performed when tasks perform input/output operations, when a
		task indicates that it can be switched, or on hardware interrupts.}
	\end{quotation}
	Next step in our awesome-kernel, execution.\\
	And as you can read above, we need some structures, interface. So before we're
	gettin' all dirty working on binary, let's code those.\\
	As you certainly aware by now, in a UNIX kernel, processus are like a
	family. Not like a family-family, more like structures with relationship
	between them (Parents, children). Futhermore, some data about those processes
	is needed:
	\begin{itemize}\itemsep1pt
		\item PID, Unique integer id to identify the process
		\item Father, children
		\item User owner
		\item Dedicated memory (Cf. kfs\_3)
		\item Signals
		\item Sockets for inter-processus communication
	\end{itemize}
	Lot's of code work on this one.

\newpage
%******************************************************************************%
%                                                                              %
%                                  Goals                                       %
%                                                                              %
%                                                                              %
%******************************************************************************%
\chapter{Goals}

At the end of this project, you will added the following to your kernel:
\begin{itemize}\itemsep1pt
	\item Basic data structure for processus
	\item Processus interconnection, such as kinship, signals and sockets.
	\item Processus owner
	\item Rights on processus
	\item Helpers for the followings syscalls: \texttt{fork}, \texttt{wait}, 
	\texttt{\_exit}, \texttt{getuid}, \texttt{signal}, \texttt{kill}
	\item Processus interruptions
	\item Processus memory separation
	\item Multitasking
\end{itemize}


\newpage
%******************************************************************************%
%                                                                              %
%                             General instructions                             %
%                                                                              %
%******************************************************************************%
\chapter{General instructions}
	\section{Code and Execution}
		\subsection{Emulation}
		The following part is not mandatory, you're free to use any virtual
		manager you want to, however, i suggest you to use \texttt{KVM}.
		It's a \texttt{Kernel Virtual Manager}, and have advanced execution
		and debugs functions.
		All of the example below will use \texttt{KVM}.
		\subsection{Language}
			The \texttt{C} language is not mandatory, you can use any language
			you want for this suit of projects.\\
			Keep in mind that all language are not kernel friendly, you could
			code a kernel with \texttt{Javascript}, but are you sure
			it's a good idea ?\\ Also, a lot of the documentation are
			in \texttt{C}, you will have to 'translate' the code all along
			if you choose a different language.\\

			Furthermore, all of the features of a language cannot be used in a
			basic kernel. Let's take an example with \texttt{C++} :\\
			This language uses 'new' to make allocation, class and structures
			declaration. But in your kernel, you don't have a memory interface
			(yet), so you can't use those features now.\\

			A lot of language can be used instead of \texttt{C},
			like \texttt{C++}, \texttt{Rust}, \texttt{Go}, etc.
			You can even code your entire kernel in \texttt{ASM} !\\
			\begin{center}
			  \includegraphics[width=8cm]{choose.jpg}
			\end{center}

\newpage


	\section{Compilation}
		\subsection{Compilers}
			You can choose any compilers you want. I personnaly use	\texttt{gcc}
			and \texttt{nasm}. A Makefile must be turn in to.
		\subsection{Flags}
			In order to boot your kernel without any dependencies, you must compile
			your code with the following flags (Adapt the flags for your language,
			those ones are a \texttt{C++} example):
			\begin{itemize}\itemsep1pt
				\item \texttt{-fno-builtin}
				\item \texttt{-fno-exception}
				\item \texttt{-fno-stack-protector}
				\item \texttt{-fno-rtti}
				\item \texttt{-nostdlib}
				\item \texttt{-nodefaultlibs}
			\end{itemize}
			Pay attention to \texttt{-nodefaultlibs} and \texttt{-nostdlib}.
			Your Kernel will be compiled on a host system, yes, but cannot be
			linked to any existing library on that host, otherwise it will not
			be executed.
	\section{Linking}
		You cannot use an existing linker in order to link your kernel.
		As written above, your kernel will not boot. So, you must create a linker
		for your kernel.\\
		Be carefull, you \texttt{CAN} use the 'ld' binary available on your host,
		but you \texttt{CANNOT} use the .ld file of your host.
	\section{Architecture}
		The \texttt{i386} (x86) architecture is mandatory
		(you can thank me later).
	\section{Documentation}
		There is a lot of documentation available, good and bad.
		I personnaly think the \texttt{\href{http://wiki.osdev.org/Main_Page}
		{OSDev}} wiki is one of the best.
	\section{Base code}
		In this subject, you have to take your precedent \texttt{KFS} code,
		and work from it !\\ Or don't. And rewrite all from scratch. Your call !

\newpage
%******************************************************************************%
%                                                                              %
%                             Mandatory part                                   %
%                                                                              %
%******************************************************************************%
\chapter{Mandatory part}

	\section{Data}
	You will need to implement a complete interface for processes in your kernel.\\
	Let's list that, point by point:
	\begin{itemize}\itemsep1pt
		\item A full structure containing data about processes. That includes:
		\begin{itemize}\itemsep1pt
			\item A PID.
			\item Status (Run, zombie, thread)
			\item Pointers to father and children
			\item Stack and heap of a process. (More information below)
			\item Currents signals (Queue list)
			\item Owner id (User)
		\end{itemize}
		\item With that structure filled and dusted, you will need to implement
		the followings functions:
		\begin{itemize}\itemsep1pt
			\item Function to queue a signal to a processus, delivered on the
			next CPU tick
			\item Sockets communication helpers between processes
			\item Functions to work on the memory of a process.
			\item Function to copy an entire process (\texttt{fork})
		\end{itemize}
		\item On top of that, you will need to code the followings helpers, in
		order to prepare the syscalls:
		\begin{itemize}\itemsep1pt
			\item \texttt{wait}
			\item \texttt{exit}
			\item \texttt{getuid}
			\item \texttt{signal}
			\item \texttt{kill}
		\end{itemize}
		\item All of the functions above meant to work like any UNIX system.
	\end{itemize}
	\section{Memory}
	Some note about the process memory.\\
	In kfs\_3, you have been implementing a paging memory. Now's the time to use it !
	Set a page size, then when a process is created, assign one page to him.
	Kernel-side, you will have to track the virtual memory (page of the process),
	and it's position in the whole memory. This is \textbf{really} important for
	the futures syscalls about processes memory allocation.
	\section{Execution}
	Your kernel \textbf{must} handle multitasking. There are many ways to do it,
	you must choose solution to implement that. Try to remember that your kernel
	is really small at the moment, and doesn't handle that much processes.
	\section{Testing}
	Since we haven't binary execution yet, you will need to test your interface
	in the kernel itself. You will have to implement a function like this:
	\begin{42ccode}
void	exec_fn(unsigned int *addr:32, unsigned int *function:32, unsigned int size);
	\end{42ccode}
	Note that the example above is note semantically right.\\
	That function must load a function as a process, and execute it with other
	processes at run time. You will have to prove your work from that function,
	be creative !


\newpage
%******************************************************************************%
%                                                                              %
%                                 Bonus part                                   %
%                                                                              %
%******************************************************************************%
\chapter{Bonus part}
	Before reading further, remember that everything above must be \textbf{perfect}
	in order to grade the bonuses. And when i mean perfect, i mean \textbf{perfect}.\\
	Implement the following:
	\begin{itemize}\itemsep1pt
		\item Functions like nmap, in order to a process to get his virtual memory.
		\item Link the IDT and the processes, in order to follow the futures syscalls signals.
		\item Create the BSS and data sectors in the process structure.
	\end{itemize}


\newpage
%******************************************************************************%
%                                                                              %
%                           Turn-in and peer-evaluation                        %
%                                                                              %
%******************************************************************************%
\chapter{Turn-in and peer-evaluation}

	Turn your work into your \texttt{GiT} repository, as usual.
	Only the work present on your repository will be graded in defense.\\

	Your must turn in your code, a Makefile and a basic virtual image for your kernel.\\
	Side note about that image, your kernel does nothing with it yet,
	SO THERE IS NO NEED TO BE SIZED LIKE AN ELEPHANT.




%******************************************************************************%
\end{document}
