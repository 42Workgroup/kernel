%******************************************************************************%
% Copyright (C) 2018  Louis Solofrizzo                                         %
%                                                                              %
% This content is considered a free software: you can redistribute it          %
% and/or modify it under the terms of the GNU General Public License as        %
% published by the Free Software Foundation, either version 3 of the License,  %
% or (at your option) any later version.                                       %
%                                                                              %
% This program is distributed in the hope that it will be useful,              %
% but WITHOUT ANY WARRANTY; without even the implied warranty of               %
% MERCHANTABILITY or FITNESS FOR A PARTICULAR PURPOSE.  See the                %
% GNU General Public License for more details.                                 %
%                                                                              %
% You should have received a copy of the GNU General Public License            %
% along with this program.  If not, see <https://www.gnu.org/licenses/>.       %
%******************************************************************************%

%******************************************************************************%
%                                                                              %
%                           KFS_2.en.tex for KFS_2                             %
%                                                                              %
%                  Created on : Wed May 26 13:37:00 2016                       %
%          Made by : Louis "Ne02ptzero" Solofrizzo <louis@ne02ptzero.me>       %
%                                                                              %
%******************************************************************************%

\documentclass{42-en}


%******************************************************************************%
%                                                                              %
%                                    Header                                    %
%                                                                              %
%******************************************************************************%
\begin{document}



                           \title{KFS\_2}
                          \subtitle{GDT \& Stack}
                       \member{Louis Solofrizzo}{louis@ne02ptzero.me}
                        \member{42 Staff}{pedago@42.fr}

\summary {
	Let's code the stack !
}

\maketitle

\tableofcontents

\newpage
%******************************************************************************%
%                                                                              %
%                                  Forewords                                   %
%                                                                              %
%******************************************************************************%
\chapter{Forewords}
	\subsection{Hunter 6.2.4 Rotation}
		\begin{itemize}\itemsep1pt
			\item Barrage
			\item Aspect of the pack
			\item Die
			\item Blame the heal
		\end{itemize}
	\subsection{Rogue 6.2.4 Rotation}
		\begin{itemize}\itemsep1pt
			\item Pull without the tank
			\item Die
			\item Blame the heal
		\end{itemize}
	\subsection{Warrior tank 6.2.4 Rotation}
		\begin{itemize}\itemsep1pt
			\item Pull the entire dungeon
			\item Die
			\item Blame the heal
		\end{itemize}
	\subsection{Druid Heal 6.2.4 Rotation}
		\begin{itemize}\itemsep1pt
			\item Blame the other heal
		\end{itemize}
	\subsection{Death Knight 6.2.4 Rotation}
		\begin{itemize}\itemsep1pt
			\item Roll
			\item Your
			\item Head
			\item On
			\item The
			\item Keyboard
		\end{itemize}

\newpage
%******************************************************************************%
%                                                                              %
%                                 Introduction                                 %
%                                                                              %
%******************************************************************************%
\chapter{Introduction}

	Welcome in \texttt{Kernel from Scratch}, second subject.
	This time, you will code a stack, and integrate it with the \texttt{GDT}.\\
	\\
	What is a \texttt{GDT} ? Let's see:\\
	\\
	The \texttt{GDT} ("Global Descriptor Table") is a data structure used to define the
	different memory areas: the base address, the size and access privileges
	like execute and write. These memory areas are called "segments".\\
	\\In a \texttt{GDT}, you can find:
	\begin{itemize}\itemsep1pt
		\item Kernel code, used to store the executable binary code
		\item Kernel data
		\item Kernel stack, used to store the call stack during kernel execution
		\item User code, used to store the executable binary code for user programs
		\item User program data
		\item User stack, used to store the call stack during execution in userland
	\end{itemize}
	I think you can see why this thing is really important in a Kernel !\\
	\\
	Next, the stack. I'm sure you all know what a stack is, but here's a friendy
	reminder:
	\begin{quotation}
		\textit{In computer science, a stack is an abstract data type that serves as a
		collection of elements, with two principal operations: push, which adds an
		element to the collection, and pop, which removes the most recently added
		element that was not yet removed. The order in which elements come off a
		stack gives rise to its alternative name, LIFO (for last in, first out).}
	\end{quotation}
	Get it ? Good. Now, let's move on.

\newpage
%******************************************************************************%
%                                                                              %
%                                  Goals                                       %
%                                                                              %
%******************************************************************************%
\chapter{Goals}


	In this subject, you will have to \texttt{create}, \texttt{fill} and \texttt{link}
	a Global Descriptor Table into your Kernel.\\
	\\
	Yup, that's it. Not so much, eh ?\\
	\\
	Actually, you will have to understand how "memory" really works in a system,
	how the Stack and RAM works, how to use it, how to fill it and how to link it
	with the \texttt{BIOS}.\\
	\\
	Yeah, the \texttt{BIOS}. Thanks to \texttt{GRUB}, it will help you a lot !\\
	Good thing you already installed it.

\newpage
%******************************************************************************%
%                                                                              %
%                             General instructions                             %
%                                                                              %
%******************************************************************************%
\chapter{General instructions}
	\section{Code and Execution}
		\subsection{Emulation}
		The following part is not mandatory, you're free to use any virtual
		manager you want to, however, i suggest you to use \texttt{KVM}.
		It's a \texttt{Kernel Virtual Manager}, and have advanced execution
		and debugs functions.
		All the example below will use \texttt{KVM}.
		\subsection{Language}
			The \texttt{C} language is not mandatory, you can use any language
			you want for this suit of projects.\\
			Keep in mind that all language are not kernel friendly, you could
			code a kernel with \texttt{Javascript}, but are you sure
			it's a good idea ?\\ Also, a lot of the documentation are
			in \texttt{C}, you will have to 'translate' the code all along
			if you choose a different language.\\

			Furthermore, all of the features of a language cannot be used in a
			basic kernel. Let's take an example with \texttt{C++} :\\
			This language uses 'new' to make allocation, class and structures
			declaration. But in your kernel, you don't have a memory interface
			(yet), so you can't use those features now.\\

			A lot of language can be used instead of \texttt{C},
			like \texttt{C++}, \texttt{Rust}, \texttt{Go}, etc.
			You can even code your entire kernel in \texttt{ASM} !\\
			\begin{center}
			  \includegraphics[width=8cm]{choose.jpg}
			\end{center}

\newpage

	\section{Compilation}
		\subsection{Compilers}
			You can choose any compilers you want. I personnaly use	\texttt{gcc}
			and \texttt{nasm}. A Makefile must be added to your repo.
		\subsection{Flags}
			In order to boot your kernel without any dependencies, you must compile
			your code with the following flags (Adapt the flags for your language,
			those ones are for \texttt{C++}, for instance):
			\begin{itemize}\itemsep1pt
				\item \texttt{-fno-builtin}
				\item \texttt{-fno-exception}
				\item \texttt{-fno-stack-protector}
				\item \texttt{-fno-rtti}
				\item \texttt{-nostdlib}
				\item \texttt{-nodefaultlibs}
			\end{itemize}
			Pay attention to \texttt{-nodefaultlibs} and \texttt{-nostdlib}.
			Your Kernel will be compiled on a host system, yes, but cannot be
			linked to any existing library on that host, otherwise it will not
			be executed.
	\section{Linking}
		You cannot use an existing linker in order to link your kernel.
		As written above, your kernel will not boot. So, you must create a linker
		for your kernel.\\
		Be carefull, you \texttt{CAN} use the 'ld' binary available on your host,
		but it is \textbf{FORBIDDEN} to use the .ld file of your host.
	\section{Architecture}
		The \texttt{i386} (x86) architecture is mandatory (you can thank me later).
	\section{Documentation}
		There is a lot of documentation available, good and bad.
		I personnaly think the \texttt{\href{http://wiki.osdev.org/Main_Page}
		{OSDev}} wiki is one of the best.
	\section{Base code}
		In this subject, you have to take your precedent \texttt{KFS} code,
		and work from it !\\ Or not... and rewrite all from the beginning. 
		Your call !

\newpage
%******************************************************************************%
%                                                                              %
%                             Mandatory part                                   %
%                                                                              %
%******************************************************************************%
\chapter{Mandatory part}

	Let's sum this up:
	\begin{itemize}\itemsep1pt
		\item You must create a \texttt{Global Descriptor Table}.
		\item Your \texttt{GDT} must contain:
		\begin{itemize}\itemsep1pt
			\item Kernel Code
			\item Kernel Data
			\item Kernel stack
			\item User code
			\item User data
			\item User stack
			\item Your work should not exceed 10 MB.
		\end{itemize}
		\item You must declare your \texttt{GDT} to the \texttt{BIOS}.
		\item The \texttt{GDT} must be set at address \texttt{0x00000800}.
	\end{itemize}
	When this is done, you have to code a tool to print the kernel stack,
	in a human-friendly way. (Tip: If you haven't made a \texttt{printk} yet,
	now is a good time !)

\newpage
%******************************************************************************%
%                                                                              %
%                                 Bonus part                                   %
%                                                                              %
%******************************************************************************%
\chapter{Bonus part}

	Assuming your keyboard work correctly in your Kernel, and you able to catch
	an entry, let's code a Shell !\\
	Not a POSIX Shell, just a minimalistic shell with a few commands, for 
	debugging purposes.\\
	\\
	For example, you could implement the print-kernel-stack-thing in this shell,
	and some other things like \texttt{reboot}, \texttt{halt} and such.\\
	\\
	Have fun !

\newpage
%******************************************************************************%
%                                                                              %
%                           Turn-in and peer-evaluation                        %
%                                                                              %
%******************************************************************************%
\chapter{Turn-in and peer-evaluation}

	Turn your work into your \texttt{GiT} repository, as usual.
	Only the work present on your repository will be graded in defense.\\

	Your must turn in your code, a Makefile and a basic virtual image for your kernel.\\
	Side note about that image, your kernel does nothing with it yet,
	SO THERE IS NO NEED TO BE SIZED LIKE AN ELEPHANT.




%******************************************************************************%
\end{document}
