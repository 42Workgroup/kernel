%******************************************************************************%
% Copyright (C) 2018  Louis Solofrizzo                                         %
%                                                                              %
% This content is considered a free software: you can redistribute it          %
% and/or modify it under the terms of the GNU General Public License as        %
% published by the Free Software Foundation, either version 3 of the License,  %
% or (at your option) any later version.                                       %
%                                                                              %
% This program is distributed in the hope that it will be useful,              %
% but WITHOUT ANY WARRANTY; without even the implied warranty of               %
% MERCHANTABILITY or FITNESS FOR A PARTICULAR PURPOSE.  See the                %
% GNU General Public License for more details.                                 %
%                                                                              %
% You should have received a copy of the GNU General Public License            %
% along with this program.  If not, see <https://www.gnu.org/licenses/>.       %
%******************************************************************************%

%******************************************************************************%
%                                                                              %
%                        KFS_10.en.tex for KFS_10                              %
%                                                                              %
%                  Created on : Wed May 25 13:27:28 2016                       %
%          Made by : Louis "Ne02ptzero" Solofrizzo <louis@ne02ptzero.me>       %
%                                                                              %
%******************************************************************************%

\documentclass{42-en}


%******************************************************************************%
%                                                                              %
%                                    Header                                    %
%                                                                              %
%******************************************************************************%
\begin{document}



                           \title{KFS\_10}
                          \subtitle{The END}
                       \member{Louis Solofrizzo}{louis@ne02ptzero.me}
                        \member{42 Staff}{pedago@42.fr}

\summary {
    The End of the world. Or of this series of projects. Can't decide.
}

\maketitle

\tableofcontents


%******************************************************************************%
%                                                                              %
%                                  Foreword                                    %
%                                                                              %
%******************************************************************************%
\chapter{Foreword}

%******************************************************************************%
%                                                                              %
%                                 Introduction                                 %
%                                                                              %
%******************************************************************************%
\chapter{Introduction}
    This is the last KFS project. You may shed manly tears.\\
    This subject is about making your kernel a complete Unix system.\\
    Nothing specific here.\\

%******************************************************************************%
%                                                                              %
%                                  Goals                                       %
%                                                                              %
%******************************************************************************%
\chapter{Goals}
    At the end of this project, you will have a complete OS. It's about god
    damn time.
    \begin{itemize}\itemsep1pt
        \item Fullu functional basic binaries /bin/*.
        \item Libc.
        \item A Posix Shell.
    \end{itemize}

%******************************************************************************%
%                                                                              %
%                             General instructions                             %
%                                                                              %
%******************************************************************************%
\chapter{General instructions}
    \section{Code and Execution}
        \subsection{Emulation}
        The following part is not mandatory, you're free to use any virtual
        manager you want; however, I suggest you use \texttt{KVM}.
        It's a \texttt{Kernel Virtual Manager} with advanced execution
        and debug functions.
        All of the examples below will use \texttt{KVM}.
        \subsection{Language}
            The \texttt{C} language is not mandatory, you can use any language
            you want for this series of projects.\\
            Keep in mind that not all languages are kernel friendly though, you
            could code a kernel in \texttt{Javascript}, but are you sure it's a
            good idea?\\
            Also, most of the documentation is written in \texttt{C}, you will
            have to 'translate' the code all along if you choose a different
            language.\\

            Furthermore, not all the features of a given language can be used
            in a basic kernel. Let's take an example with \texttt{C++}:\\
            this language uses 'new' to make allocations, classes and
            structures declarations. But in your kernel you don't have a memory
            interface (yet), so you can't use any of these features.\\

            Many languages can be used instead of \texttt{C},
            like \texttt{C++}, \texttt{Rust}, \texttt{Go}, etc.
            You can even code your entire kernel in \texttt{ASM}!\\
            So yes, you may choose a language. But choose wisely.
            \begin{center}
              \includegraphics[width=8cm]{choose.jpg}
            \end{center}
\newpage

    \section{Compilation}
        \subsection{Compilers}
            You can choose any compiler you want. I personaly use \texttt{gcc}
            and \texttt{nasm}. A Makefile must be turned-in as well.
        \subsection{Flags}
            In order to boot your kernel without any dependency, you must
            compile your code with the following flags (adapt the flags for
            your language, these are \texttt{C++} examples):
            \begin{itemize}\itemsep1pt
                \item \texttt{-fno-builtin}
                \item \texttt{-fno-exception}
                \item \texttt{-fno-stack-protector}
                \item \texttt{-fno-rtti}
                \item \texttt{-nostdlib}
                \item \texttt{-nodefaultlibs}
            \end{itemize}
            You might have noticed these two flags: \texttt{-nodefaultlibs}
            and \texttt{-nostdlib}. Your Kernel will be compiled on a host
            system, that's true, but it cannot be linked to any existing
            library on that host, otherwise it will not be executed.
    \section{Linking}
        You cannot use an existing linker in order to link your kernel.
        As mentionned above, your kernel would not be initialized. So you must
        create a linker for your kernel.\\
        Be careful, you \texttt{CAN} use the 'ld' binary available on your
        host, but you \texttt{CANNOT} use the .ld file of your host.
    \section{Architecture}
        The \texttt{i386} (x86) architecture is mandatory (you can thank
        me later).
    \section{Documentation}
        There is a lot of documentation available, good and bad.
        I personaly think the \texttt{\href{http://wiki.osdev.org/Main_Page}
        {OSDev}} wiki is one of the best.
    \section{Base code}
        In this subject, you have to take your previous \texttt{KFS} code,
        and work from it!\\
        Or don't. And rewrite everything from scratch. Your call!
\newpage

%******************************************************************************%
%                                                                              %
%                             Mandatory part                                   %
%                                                                              %
%******************************************************************************%
\chapter{Mandatory part}
    You must install the following:
    \begin{itemize}\itemsep1pt
        \item A POSIX shell. sh will do.
        \item The complete libc.
        \item Basic Unix binaries:
        \begin{itemize}\itemsep1pt
            \item cat
            \item chmod
            \item cp
            \item date
            \item dd
            \item df
            \item echo
            \item hostname
            \item kill
            \item ln
            \item ls
            \item mkdir
            \item mv
            \item ps
            \item pwd
            \item rm
            \item rmdir
            \item sleep
        \end{itemize}
    \end{itemize}


%******************************************************************************%
%                                                                              %
%                                 Bonus part                                   %
%                                                                              %
%******************************************************************************%
\chapter{Bonus part}
Install whatever you want. No really, I mean it, make your OS yours.\\
Have fun :)

%******************************************************************************%
%                                                                              %
%                           Turn-in and peer-evaluation                        %
%                                                                              %
%******************************************************************************%
\chapter{Turn-in and peer-evaluation}

    Turn your work in using your \texttt{GiT} repository, as
    usual. Only the work that's in your repository will be graded during
    the evaluation.

    Your must turn in your code, a Makefile and a basic virtual image for your
    kernel.\\
    Careful about that image, your kernel does nothing with it yet,
    SO THERE IS NO NEED TO BE BUILT LIKE AN ELEPHANT. (More than 10Mo is waaay
    too much.)

%******************************************************************************%
\end{document}
