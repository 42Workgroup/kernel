%******************************************************************************%
% Copyright (C) 2018  Louis Solofrizzo                                         %
%                                                                              %
% This content is considered a free software: you can redistribute it          %
% and/or modify it under the terms of the GNU General Public License as        %
% published by the Free Software Foundation, either version 3 of the License,  %
% or (at your option) any later version.                                       %
%                                                                              %
% This program is distributed in the hope that it will be useful,              %
% but WITHOUT ANY WARRANTY; without even the implied warranty of               %
% MERCHANTABILITY or FITNESS FOR A PARTICULAR PURPOSE.  See the                %
% GNU General Public License for more details.                                 %
%                                                                              %
% You should have received a copy of the GNU General Public License            %
% along with this program.  If not, see <https://www.gnu.org/licenses/>.       %
%******************************************************************************%

%******************************************************************************%
%                                                                              %
%                          KFS_8.en.tex for KFS_8                              %
%                                                                              %
%                  Created on : Wed May 25 13:27:28 2016                       %
%          Made by : Louis "Ne02ptzero" Solofrizzo <louis@ne02ptzero.me>       %
%                                                                              %
%******************************************************************************%

\documentclass{42-en}


%******************************************************************************%
%                                                                              %
%                                    Header                                    %
%                                                                              %
%******************************************************************************%
\begin{document}



                           \title{KFS\_8}
                          \subtitle{Modules}
                       \member{Louis Solofrizzo}{louis@ne02ptzero.me}
                        \member{42 Staff}{pedago@42.fr}

\summary {
    Modular Kernel
}

\maketitle

\tableofcontents


%******************************************************************************%
%                                                                              %
%                                  Foreword                                    %
%                                                                              %
%******************************************************************************%
\chapter{Foreword}

%******************************************************************************%
%                                                                              %
%                                 Introduction                                 %
%                                                                              %
%******************************************************************************%
\chapter{Introduction}
    Extract from \href{http://wiki.osdev.org/Modular_Kernel}{OSDev.}
    \subsection{Modular Kernel}
    A modular kernel is an attempt to merge the good points of kernel-level
    drivers and third-party drivers. In a modular kernel, some part of the
    system core will be located in independent files called modules that can be
    added to the system at run time. Depending on the content of those modules,
    the goal can vary such as:
    \begin{itemize}\itemsep1pt
        \item only loading drivers if a device is actually found
        \item only load a filesystem if it gets actually requested
        \item only load the code for a specific (scheduling/security/whatever)
        policy when it should be evaluated
        \item etc.
    \end{itemize}
    The basic goal remains however the same: keep what is loaded at boot-time
    minimal while still allowing the kernel to perform more complex functions.
    The basics of modular kernel are very close to what we find in
    implementation of plugins in applications or dynamic libraries in general.
    \subsection{What are some advantages and disadvantages for a Modular Kernel?}
    \textbf{Advantages}
    \begin{itemize}\itemsep1pt
        \item The most obvious is that the kernel doesn't have to load
        everything at boot time; it can be expanded as needed. This can
        decrease boot time, as some drivers won't be loaded unless the hardware
        they run is used (NOTE: This boot time decrease can be negligible
        depending on what drivers are modules, how they're loaded, etc.)
        \item The core kernel isn't as big
        \item If you need a new module, you don't have to recompile.
    \end{itemize}
    \textbf{Disadvantages}
    \begin{itemize}\itemsep1pt
        \item It may lose stability. If there is a module that does something
        bad, the kernel can crash, as modules should have full permissions.
        \item ...and therefore security is compromised. A module can do
        anything, so one could easily write an evil module to crash things.
        (Some OSs, like Linux, only allow modules to be loaded by the root
        user.)
        \item Coding can be more difficult, as the module cannot reference
        kernel procedures without kernel symbols.
    \end{itemize}
    \subsection{ What does a Modular Kernel look like ?}
    There are several components that can be identified in virtually every
    modular kernel:
    \begin{itemize}\itemsep1pt
        \item \textbf{The core}: this is the collection of features in the
        kernel that are absolutely mandatory regardless of whether you have
        modules or not.
        \item \textbf{The modules loader}: this is a part of the system that
        will be responsible of preparing a module file so that it can be used
        as if it was a part of the core itself.
        \item \textbf{The kernel symbols Table}: This contains additional
        information about the core and loaded modules that the module loader
        needs in order to link a new module to the existing kernel.
        \item \textbf{The dependencies tracking}: As soon as you want to unload
        some module, you'll have to know whether you can do it or not.
        Especially, if a module X has requested symbols from module Z, trying
        to unload Z while X is present in the system is likely to cause havoc.
        \item \textbf{Modules}: Every part of the system you might want
        (or don't want) to have.
    \end{itemize}

%******************************************************************************%
%                                                                              %
%                                  Goals                                       %
%                                                                              %
%******************************************************************************%
\chapter{Goals}
    At the end of this project, you will have a full kernel module interface.
    That entails:
    \begin{itemize}\itemsep1pt
        \item Registering kernel modules (Creation / Destruction).
        \item Loading modules at boot time.
        \item Implementing functions for communication / callback between
        the kernel and the modules.
    \end{itemize}

%******************************************************************************%
%                                                                              %
%                             General instructions                             %
%                                                                              %
%******************************************************************************%
\chapter{General instructions}
    \section{Code and Execution}
        \subsection{Emulation}
        The following part is not mandatory, you're free to use any virtual
        manager you want; however, I suggest you use \texttt{KVM}.
        It's a \texttt{Kernel Virtual Manager} with advanced execution
        and debug functions.
        All of the examples below will use \texttt{KVM}.
        \subsection{Language}
            The \texttt{C} language is not mandatory, you can use any language
            you want for this series of projects.\\
            Keep in mind that not all languages are kernel friendly, you could
            code a kernel in \texttt{Javascript}, but are you sure it's a
            good idea?\\
            Also, most of the documentation is written in \texttt{C}, you will
            have to 'translate' the code all along if you choose a different
            language.\\

            Furthermore, not all the features of a given language can be used
            in a basic kernel. Let's take an example with \texttt{C++}:\\
            this language uses 'new' to make allocations, classes and
            structures declarations. But in your kernel you don't have a memory
            interface (yet), so you can't use any of these features.\\

            Many languages can be used instead of \texttt{C},
            like \texttt{C++}, \texttt{Rust}, \texttt{Go}, etc.
            You can even code your entire kernel in \texttt{ASM}!\\
            \begin{center}
              \includegraphics[width=8cm]{choose.jpg}
            \end{center}
\newpage

    \section{Compilation}
        \subsection{Compilers}
            You can choose any compiler you want. I personaly use \texttt{gcc}
            and \texttt{nasm}. A Makefile must be turned-in as well.
        \subsection{Flags}
            In order to boot your kernel without any dependency, you must
            compile your code with the following flags (adapt the flags for
            your language, these are \texttt{C++} examples):
            \begin{itemize}\itemsep1pt
                \item \texttt{-fno-builtin}
                \item \texttt{-fno-exception}
                \item \texttt{-fno-stack-protector}
                \item \texttt{-fno-rtti}
                \item \texttt{-nostdlib}
                \item \texttt{-nodefaultlibs}
            \end{itemize}
            You might have noticed these two flags: \texttt{-nodefaultlibs}
            and \texttt{-nostdlib}. Your Kernel will be compiled on a host
            system, that's true, but it cannot be linked to any existing
            library on that host, otherwise it will not be executed.
    \section{Linking}
        You cannot use an existing linker in order to link your kernel.
        As mentionned above, your kernel would not be initialized. So you must
        create a linker for your kernel.\\
        Be careful, you \texttt{CAN} use the 'ld' binary available on your
        host, but you \texttt{CANNOT} use the .ld file of your host.
    \section{Architecture}
        The \texttt{i386} (x86) architecture is mandatory (you can thank
        me later).
    \section{Documentation}
        There is a lot of documentation available, good and bad.
        I personaly think the \texttt{\href{http://wiki.osdev.org/Main_Page}
        {OSDev}} wiki is one of the best.
    \section{Base code}
        In this subject, you have to take your previous \texttt{KFS} code,
        and work from it!\\
        Or don't. And rewrite everything from scratch. Your call!
\newpage

%******************************************************************************%
%                                                                              %
%                             Mandatory part                                   %
%                                                                              %
%******************************************************************************%
\chapter{Mandatory part}
    In this subject, you will have to:
    \begin{itemize}\itemsep1pt
        \item Write an internal API to manage modules integration.
        \item Handle the creation and destruction of the modules.
        \item Write functions to communicate between the kernel and the module.
        \item Write callbacks between the kernel and the module.
        \item Make these callbacks configurable. For example, if a module wants
        a callback function called at every CPU cycle, it must declare it so
        that the kernel calls the function when the module is created.
        \item Give these modules access to the kernel functions for memory,
        sockets, process, etc.
    \end{itemize}
    In order to prove that your work is functional, you must implement 2
    modules: a keyboard module and a time module.\\
    \\
    \textbf{Keyboard Module}\\
    The keyboard module must get a callback from the Kernel every time a
    key is pressed / realeased.\\
    \textbf{Time Module}\\
    The time module must return a struct / value every time the kernel asks for
    it via function / ioctl / callback.

%******************************************************************************%
%                                                                              %
%                                 Bonus part                                   %
%                                                                              %
%******************************************************************************%
\chapter{Bonus part}
    Create special memory allocation functions, in order to create a memory
    ring dedicated to the modules.


%******************************************************************************%
%                                                                              %
%                           Turn-in and peer-evaluation                        %
%                                                                              %
%******************************************************************************%
\chapter{Turn-in and peer-evaluation}

    Turn your work in using your \texttt{GiT} repository, as
    usual. Only the work that's in your repository will be graded during
    the evaluation.

    Your must turn in your code, a Makefile and a basic virtual image for your
    kernel.\\
    Side note about this image,
    THERE IS NO NEED TO BE BUILT LIKE AN ELEPHANT.


%******************************************************************************%
\end{document}
